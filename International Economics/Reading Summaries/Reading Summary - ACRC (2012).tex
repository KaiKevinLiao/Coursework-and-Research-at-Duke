\documentclass{article}
\usepackage[utf8]{inputenc}
\usepackage{amsmath}
\usepackage{geometry}
\newcommand*\diff{\mathop{}\!\mathrm{d}}
\newcommand*\Diff[1]{\mathop{}\!\mathrm{d^#1}}
\usepackage{bbm}
\geometry{left=3.5cm,right=3.5cm,top=2cm,bottom=2.5cm}

\title{Reading Summary - ACRC (2012)}
\author{kai.liao }
\date{February 2020}

\begin{document}
\maketitle
Arkolakis, Costinot and Rodriguez-Clare study the welfare gains from international trade. They find that in a large class of trade models we studied in class (E-K, Krugman, and Melitz-Chaney), welfare gains from international trade can be summarized by a unified welfare measure. This paper basically saying that we can "skip" the different structural \\

There are several primitive assumptions of international trade theories we studied in class.\\
1. Dixit-Stiglitz preference. \\
2. Linear cost function. The cost of a firm can be pinned down by firm specific productivity together with country specific wages, fixed costs, marginal export costs, and fixed export costs.\\
3. Perfect or monopolistic competition. Whether the entry is free or not does not change our measure of welfare.\\
4. Bilateral balanced trade\\
5. A CES import demand system.\\

The welfare is measured by $W_j = \frac{R_j}{P_j}$. We are interested in effect of foreign shock to domestic welfare. Foreign socks includes  any changes in labor endowments, entry costs, variable trade costs, and fixed trade costs that do not affect either domestic
endowment or its ability to serve its own market.

The "simple" proposition of welfare is given by,
\begin{equation}
    \hat{W} = \hat{\lambda^{1/\epsilon}}
\end{equation}
where $\hat{W} = \frac{W'}{W}$ is the change in welfare, and $\hat{\lambda} = \frac{\lambda'}{\lambda}$ is the change in the share of domestic expenditure.\\

They proved this proposition in their Appendix 1. The basic idea of the proof is to write different equations for small changes in price index ($\diff\ln{w_j}$).\\

Another way of interpreting this results is the whenever an international trade theory that assumes the same primitive restrictions as this paper, their measure of welfare change can be characterized as proposition (1). In another word, this paper gives a sufficient set of assumptions set for this proposition. Under these assumptions, there exist two
sufficient statistics for welfare analysis: (i) the share of expenditure on domestic
goods; and (ii) the trade elasticity.\\

The remaining parts of this paper uses the three macro-level restrictions, and infer changes
in real income from changes in aggregate trade flows alone in Eaton-Kortum and Melitz-Chaney model, thereby leading to the
same welfare formula as in a simple Armington-Krugman model.\\

A important remarks made by this paper is that should not be interpreted as a negative result. Having
welfare gains in different models summarized by the same statistics
does not mean the magnitude of those gains is the same in all
models.
In reality, the elasticity of trade is
not observed in the data and it is dependent to the specific model and the empirical specification that we use.



\end{document}
