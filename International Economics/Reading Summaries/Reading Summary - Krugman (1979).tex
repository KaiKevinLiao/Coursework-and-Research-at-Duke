\documentclass{article}
\usepackage[utf8]{inputenc}
\usepackage{amsmath}
\usepackage{amssymb}
\usepackage{geometry}
\usepackage{titlesec}
\usepackage{makecell}
\geometry{left=2.5cm,right=2cm,top=2cm,bottom=2.5cm}


\title{Reading Summary - Krugman (1979)}
\author{kai.liao }
\date{February 2020}

\begin{document}

\maketitle

Krugman (1979) develops a general equilibrium model of noncomparative advantage trade. While the Ricardian model attribute trade to comparative advantage of production and Heckscher-Ohlin attribute trade to different endowments, Krugman states that trade can be driven by economies of scale. 
\section{Evaluate the Assumptions}
\begin{itemize}
  \item Krugman assumes increasing return to scale which is driven by constant marginal costs and a fixed cost. 
  $$l_i = \alpha + \beta x_i$$
  Compared with Heckscher-Olhlin's assumption of a decreasing return to scale, Krugman's assumption is closer to reality. 
  \item The monopolistic competition comes after the assumption of IRS. This is a better resemblance to the real world than perfect competition. With free exit and enter, the long run equilibrium wipes out excess profits. 
  \item The utility function of all residents are assumed to be same as follow
  $$U = \sum_{i = 1}^n v(c_i) \text{~~~~~~} v' > 0\text{,~~}v''<0$$
  This assumption shows that consumers like variety. For example, assume $\sum_{i = 0}^n c_i$ is constant and \newline $c_i = \frac{1}{n}\sum_{i = 0}^n c_i$. $U$ is monotonically increasing with respect to $n$. This assumption is important for showing gains from international trade, because foreign countries provide more choice for consumers (increase n). 
  
  \item Because of symmetric, $n$ is decided by the following equation,
  $$n = \frac{L}{\alpha + \beta x}$$
  where $L$ is the total labor, measured by size of economy.
  

\end{itemize}
\section{Thoughts}
This paper is the first to discuss monopolistic competition and trade. Therefore there are many points that can be extended.
\begin{itemize}
    \item Firms exit after international trade. Once  the  home  country  enter  a  free  trade  with  a  foreign  country,  domestic  firms  face  more  competitors  and bigger potential markets. These will change the elasticity of demand and thus force some unprofitable firms to exit. As a result, the equilibrium $\Bar{n}$ is smaller than $n+n^*$, and further research shows that the international trade will increase productivity because more intense competition makes firms with lower productivity exit.
    \item Further research shows the relation between monopolistic competition and inter-industry trade.
\end{itemize}

\end{document}
